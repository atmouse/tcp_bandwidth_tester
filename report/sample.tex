%===============================================================================
% sample.tex
% Sample structure for RW344 essays.
%===============================================================================

\documentclass[11pt,a4paper,twocolumn]{article}
\usepackage{a4wide}
\renewcommand{\baselinestretch}{1.1}
\setlength\overfullrule{5pt}

% -- DO NOT CHANGE THE SETTINGS ABOVE --

\begin{document}
\title{
	TCP Congestion Control Comparison
}
\author{
	Arne Esterhuizen
}
\date{20 Janaury 2012}
\maketitle

\begin{abstract}
The purpose of this report is to investigate the effects that different TCP variants have on each other.
These TCP variants differ by the congestion control mechanisms they employ. Congestion control mechanisms affect
how much network traffic is generated by TCP at any one time and aims to prevent a TCP connection from over utilising
the network. This report investigates the different congestion control mechanisms that are included as loadable
modules in the Linux kernel, and discusses experiments performed on how these congestion control mechanisms compete
for network resources. It will be shown that some TCP variants are able to co-exist, whilst others tend to 
use excessive bandwidth, potentially smothering other competing TCP connections.
Should the results prove interesting, the project may be given to third year computer network students to 
perform as an assignment.
\end{abstract}

\section{Table of Contents}

\section{Introduction}
\label{sec:intro}

The \textit{Transmission Control Protocol} (TCP) controls how much data it transmits over a network by utilising
a congestion window. TCP cannot send more data than the congestion window allows. The size of TCP's congestion
window depends on the condition of the network. In the case of the network experiencing heavy data traffic, the congestion window
would be shortened. In the case of the network being only lightly loaded with data traffic, the congestion window would
gradually expand. How and when to adjust the congestion window depends on whatever form of congestion control
a TCP connection utilises. Congestion control mechanisms rely on various indicators to passively ascertain the state of the network.
Packet loss is an indication that the network is being over utilised, and that routers along the way are dropping IP packets
due to limited buffer space. Also, routers can set special flags in an IP packet's header that will inform the receiving host
that congestion is about to occur. The receiving host can then inform the the sending host to reduce its sending rate.
Other methods include closely observing packet round trip times as well as determining the queueing delay, the time packets
spend waiting in a router's buffer.
Using one or more of these methods, congestion control mechanism are able to adjust their behaviour according to whether congestion
has occurred or could occur. As will be seen, some congestion control mechanisms allow for unfair usage of network bandwidth,
while others are able to share bandwidth equally.

The different congestion control mechanisms available for use by the Linux kernel are:
\begin{itemize}
\item HTCP
\item Hybla
\item Illinois
\item LP
\item Vegas
\item Reno
\item BIC
\item Westwood
\item YeAH
\item Cubic
\item Highspeed
\item Scalable
\item Veno
\end{itemize}
A short description of these congestion control mechanisms is given section \ref{sec:tcpoverview}. Experiments were
performed to see how these congestion control mechanism compete against each other for bandwidth. A description
of these experiments is given in section \ref{sec:exp}, followed by a discussion of the results in section \ref{sec:results}.
Technical difficulties that may affect the practicality of this project as an assignment for students is described
is section \ref{sec:diff}. Section \ref{sec:conc} summarises the presented results.

\section{Overview of TCP variants}
\label{sec:tcpoverview}

\section{Experiments performed}
\label{sec:exp}

\section{Results}
\label{sec:results}

\section{Technical Difficulties}
\label{sec:diff}

\section{Conclusion}
\label{sec:conc}


\hbadness=5000
\vbadness=5000
\bibliographystyle{plain}
\bibliography{sample}

\end{document}

%===============================================================================
% End of sample.tex
%===============================================================================

