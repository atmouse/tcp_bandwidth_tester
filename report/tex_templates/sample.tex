%===============================================================================
% sample.tex
% Sample structure for RW344 essays.
%===============================================================================

\documentclass[11pt,a4paper,twocolumn]{article}
\usepackage{a4wide}
\renewcommand{\baselinestretch}{1.1}
\setlength\overfullrule{5pt}

% -- DO NOT CHANGE THE SETTINGS ABOVE --

\begin{document}
\title{
	Essay \#1: The Gods of Mars
}
\author{
	Edgar Rice Burroughs \\
	SNR: 87654321
}
\date{21 Apr 2013}
\maketitle

\begin{abstract}
The text in this sample comes from a novel by Edgar Rice Burroughs.  It has
nothing to do with Software Engineering and only serves as a guide to the word
count.  Usually, the abstract should give a short summary of the entire essay.
About fifty words is a good length.
\end{abstract}

\section{Introduction}

Twelve years had passed since I had laid the body of my great-uncle, Captain
John Carter, of Virginia, away from the sight of men in that strange mausoleum
in the old cemetery at Richmond.

Often had I pondered on the odd instructions he had left me governing the
construction of his mighty tomb, and especially those parts which directed that
he be laid in an OPEN casket and that the ponderous mechanism which controlled
the bolts of the vault's huge door be accessible ONLY FROM THE INSIDE.

Twelve years had passed since I had read the remarkable manuscript of this
remarkable man; this man who remembered no childhood and who could not even
offer a vague guess as to his age; who was always young and yet who had dandled
my grandfather's great-grandfather upon his knee; this man who had spent ten
years upon the planet Mars; who had fought for the green men of Barsoom and
fought against them; who had fought for and against the red men and who had won
the ever beautiful Dejah Thoris, Princess of Helium, for his wife, and for
nearly ten years had been a prince of the house of Tardos Mors, Jeddak of
Helium~\cite{Basten2004}.

\section{Finding his body}

Twelve years had passed since his body had been found upon the bluff before his
cottage overlooking the Hudson, and oft-times during these long years I had
wondered if John Carter were really dead, or if he again roamed the dead sea
bottoms of that dying planet; if he had returned to Barsoom to find that he had
opened the frowning portals of the mighty atmosphere plant in time to save the
countless millions who were dying of asphyxiation on that far-gone day that had
seen him hurtled ruthlessly through forty-eight million miles of space back to
Earth once more.  I had wondered if he had found his black-haired Princess and
the slender son he had dreamed was with her in the royal gardens of Tardos
Mors, awaiting his return.

Or, had he found that he had been too late, and thus gone back to a living
death upon a dead world?  Or was he really dead after all, never to return
either to his mother Earth or his beloved Mars?

\subsection{The telegram}

Thus was I lost in useless speculation one sultry August evening when old Ben,
my body servant, handed me a telegram.  Tearing it open I read:

\begin{center}
Meet me to-morrow hotel Raleigh Richmond.

'JOHN CARTER'
\end{center}

Early the next morning I took the first train for Richmond and within two hours
was being ushered into the room occupied by John Carter.

As I entered he rose to greet me, his old-time cordial smile of welcome
lighting his handsome face~\cite{Clarke2000}.  Apparently he had not aged a minute, but was still
the straight, clean-limbed fighting-man of thirty.  His keen grey eyes were
undimmed, and the only lines upon his face were the lines of iron character and
determination that always had been there since first I remembered him, nearly
thirty-five years before.

``Well, nephew,'' he greeted me, ``do you feel as though you were seeing a
ghost, or suffering from the effects of too many of Uncle Ben's juleps?''

``Juleps, I reckon,'' I replied, ``for I certainly feel mighty good; but maybe
it's just the sight of you again that affects me.  You have been back to Mars?
Tell me.  And Dejah Thoris?  You found her well and awaiting you?''

``Yes, I have been to Barsoom again, and---but it's a long story, too long to
tell in the limited time I have before I must return.  I have learned the
secret, nephew, and I may traverse the trackless void at my will, coming and
going between the countless planets as I list; but my heart is always in
Barsoom, and while it is there in the keeping of my Martian Princess, I doubt
that I shall ever again leave the dying world that is my life.

``I have come now because my affection for you prompted me to see you once more
before you pass over for ever into that other life that I shall never know, and
which though I have died thrice and shall die again to-night, as you know
death, I am as unable to fathom as are you.

``Even the wise and mysterious therns of Barsoom, that ancient cult which for
countless ages has been credited with holding the secret of life and death in
their impregnable fastnesses upon the hither slopes of the Mountains of Otz,
are as ignorant as we.  I have proved it, though I near lost my life in the
doing of it; but you shall read it all in the notes I have been making during
the last three months that I have been back upon Earth.''

\subsection{His notes}

He patted a swelling portfolio that lay on the table at his elbow.

``I know that you are interested and that you believe, and I know that the
world, too, is interested, though they will not believe for many years; yes,
for many ages, since they cannot understand.  Earth men have not yet progressed
to a point where they can comprehend the things that I have written in those
notes.

``Give them what you wish of it, what you think will not harm them, but do not
feel aggrieved if they laugh at you.''

That night I walked down to the cemetery with him.  At the door of his vault he
turned and pressed my hand.

``Good-bye, nephew,'' he said.  ``I may never see you again, for I doubt that I
can ever bring myself to leave my wife and boy while they live, and the span of
life upon Barsoom is often more than a thousand years.''

He entered the vault.  The great door swung slowly to.  The ponderous bolts
grated into place.  The lock clicked.  I have never seen Captain John Carter,
of Virginia, since.

\section{The return}

But here is the story of his return to Mars on that other occasion, as
I have gleaned it from the great mass of notes which he left for me
upon the table of his room in the hotel at Richmond.

There is much which I have left out; much which I have not dared to
tell; but you will find the story of his second search for Dejah
Thoris, Princess of Helium, even more remarkable than was his first
manuscript which I gave to an unbelieving world a short time since and
through which we followed the fighting Virginian across dead sea
bottoms under the moons of Mars.

As I stood upon the bluff before my cottage on that clear cold night in
the early part of March, 1886, the noble Hudson flowing like the grey
and silent spectre of a dead river below me, I felt again the strange,
compelling influence of the mighty god of war, my beloved Mars, which
for ten long and lonesome years I had implored with outstretched arms
to carry me back to my lost love.

Not since that other March night in 1866, when I had stood without that
Arizona cave in which my still and lifeless body lay wrapped in the
similitude of earthly death had I felt the irresistible attraction of
the god of my profession.

With arms outstretched toward the red eye of the great star I stood
praying for a return of that strange power which twice had drawn me
through the immensity of space, praying as I had prayed on a thousand
nights before during the long ten years that I had waited and hoped.

Suddenly a qualm of nausea swept over me, my senses swam, my knees gave
beneath me and I pitched headlong to the ground upon the very verge of
the dizzy bluff.

Instantly my brain cleared and there swept back across the threshold of
my memory the vivid picture of the horrors of that ghostly Arizona
cave; again, as on that far-gone night, my muscles refused to respond
to my will and again, as though even here upon the banks of the placid
Hudson, I could hear the awful moans and rustling of the fearsome thing
which had lurked and threatened me from the dark recesses of the cave,
I made the same mighty and superhuman effort to break the bonds of the
strange anaesthesia which held me, and again came the sharp click as of
the sudden parting of a taut wire, and I stood naked and free beside
the staring, lifeless thing that had so recently pulsed with the warm,
red life-blood of John Carter.

With scarcely a parting glance I turned my eyes again toward Mars,
lifted my hands toward his lurid rays, and waited.

Nor did I have long to wait; for scarce had I turned ere I shot with
the rapidity of thought into the awful void before me.  There was the
same instant of unthinkable cold and utter darkness that I had
experienced twenty years before, and then I opened my eyes in another
world, beneath the burning rays of a hot sun, which beat through a tiny
opening in the dome of the mighty forest in which I lay.

The scene that met my eyes was so un-Martian that my heart sprang to my
throat as the sudden fear swept through me that I had been aimlessly
tossed upon some strange planet by a cruel fate.

Why not?  What guide had I through the trackless waste of
interplanetary space?  What assurance that I might not as well be
hurtled to some far-distant star of another solar system, as to Mars?

I lay upon a close-cropped sward of red grasslike vegetation, and about
me stretched a grove of strange and beautiful trees, covered with huge
and gorgeous blossoms and filled with brilliant, voiceless birds.  I
call them birds since they were winged, but mortal eye ne'er rested on
such odd, unearthly shapes.

The vegetation was similar to that which covers the lawns of the red
Martians of the great waterways, but the trees and birds were unlike
anything that I had ever seen upon Mars, and then through the further
trees I could see that most un-Martian of all sights--an open sea, its
blue waters shimmering beneath the brazen sun.

\section{Exploring Mars}

As I rose to investigate further I experienced the same ridiculous
catastrophe that had met my first attempt to walk under Martian
conditions.  The lesser attraction of this smaller planet and the
reduced air pressure of its greatly rarefied atmosphere, afforded so
little resistance to my earthly muscles that the ordinary exertion of
the mere act of rising sent me several feet into the air and
precipitated me upon my face in the soft and brilliant grass of this
strange world.

This experience, however, gave me some slightly increased assurance
that, after all, I might indeed be in some, to me, unknown corner of
Mars, and this was very possible since during my ten years' residence
upon the planet I had explored but a comparatively tiny area of its
vast expanse.

I arose again, laughing at my forgetfulness, and soon had mastered once
more the art of attuning my earthly sinews to these changed conditions.

As I walked slowly down the imperceptible slope toward the sea I could
not help but note the park-like appearance of the sward and trees.  The
grass was as close-cropped and carpet-like as some old English lawn and
the trees themselves showed evidence of careful pruning to a uniform
height of about fifteen feet from the ground, so that as one turned his
glance in any direction the forest had the appearance at a little
distance of a vast, high-ceiled chamber.

All these evidences of careful and systematic cultivation convinced me
that I had been fortunate enough to make my entry into Mars on this
second occasion through the domain of a civilized people and that when
I should find them I would be accorded the courtesy and protection that
my rank as a Prince of the house of Tardos Mors entitled me to.

The trees of the forest attracted my deep admiration as I proceeded
toward the sea.  Their great stems, some of them fully a hundred feet
in diameter, attested their prodigious height, which I could only guess
at, since at no point could I penetrate their dense foliage above me to
more than sixty or eighty feet.

As far aloft as I could see the stems and branches and twigs were as
smooth and as highly polished as the newest of American-made pianos.
The wood of some of the trees was as black as ebony, while their
nearest neighbours might perhaps gleam in the subdued light of the
forest as clear and white as the finest china, or, again, they were
azure, scarlet, yellow, or deepest purple.

\section{Conclusion}

And in the same way was the foliage as gay and variegated as the stems,
while the blooms that clustered thick upon them may not be described in
any earthly tongue, and indeed might challenge the language of the gods.

As I neared the confines of the forest I beheld before me and between
the grove and the open sea, a broad expanse of meadow land, and as I
was about to emerge from the shadows of the trees a sight met my eyes
that banished all romantic and poetic reflection upon the beauties of
the strange landscape.

\hbadness=5000
\vbadness=5000
\bibliographystyle{plain}
\bibliography{sample}

\end{document}

%===============================================================================
% End of sample.tex
%===============================================================================

